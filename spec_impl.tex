\documentclass[12pt]{article}
\usepackage[utf8]{inputenc}
\usepackage{polski}
\usepackage{lastpage}
\usepackage{fancyhdr}

% numery strony "x / lastpage"
\pagestyle{fancy}
\fancyhf{}
\cfoot{\thepage \hspace{1pt} / \pageref{LastPage}}
% żeby też było na pierwszej stronie
\fancypagestyle{plain}{
	\fancyhf{}
	\cfoot{\thepage \hspace{1pt} / \pageref{LastPage}}
}

\title{Symulacja gry w życie Conowaya: \texttt{clife}\\
Specyfikacja implementacyjna}
\author{
	Hubert Kalbarczyk\\
	\and
	Aliaksei Samovich\\
}
\date{\today}

\begin{document}

\maketitle
\section{Istniejące katalogi}
\texttt{bin/} - katalog z plikami dla wywołania\\\\
\texttt{src/} - katalog z plikami nagłówkowymi oraz modułami projektu\\\\
\texttt{configs/} - katalog z plikami konfiguracyjnymi w TOML\\\\
\texttt{tests/} - katalog z plikami testowymi\\\\
\texttt{output/} - katalog z utworzonymi obrazkami podanych generacji oraz pliki tekstowe które zostały stworzone jako wynik gry po \textbf{n} generacjach\\\\

Plik \textbf{makefile}

\section{Moduły programu}
\subsection{Sterowanie}
 \texttt{main.c} - główny plik sterowniczy do którego łączymy inne pliki projektowe 

\subsection{Opis struktur głównych}
\begin{itemize}
    \item \texttt{struct.c (.h)} - plik ze wszystkimi strukturami w projekcie
\end{itemize}

\subsection{Alokator pamięci dla struktur}
\begin{itemize}
    \item \texttt{alloc.c (.h)} - plik w którym są funkcje alokujące pamięć dla struktur
\end{itemize}
    
\subsection{Moduły generujące}
\begin{itemize}
    \item \texttt{evolve.c (.h)} - plik w którym są funkcje symulujące grę w życie
    \item \texttt{neighbours.c (.h)} - plik z funkcjami do wyznaczania sąsiedztwa podanej komórki
    \item \texttt{border.c (.h)} - plik z funkcjami do obliczenia czy komórka wyszła za granicę
\end{itemize}

\subsection{Moduły wyjściowe}
\begin{itemize}
    \item \texttt{save2png.c (.h)} - plik z funkcjami które przetwarzają podaną planszę w obrazek \textbf{.png}
    \item \texttt{save2txt.c (.h)} - plik z funkcjami które przetwarzają podaną planszę w plik tekstowy \textbf{.txt}
    
\end{itemize}

\section{Etapy wykonania}
\textbf{1. Alokacja pamięci (alloc.c)}\\\\
\textbf{2. Czytanie konfiguracji}\\\\
\textbf{3. Generowanie i iterowanie pokoleń}\\\\
\textbf{4. Zapis wyników w pliki zgodnie z konfiguracją}\\\\
\textbf{5. Zapis wyników w pliki zgodnie z konfiguracją}\\\\

    

 \end{document}