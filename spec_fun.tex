\documentclass[12pt]{article}
\usepackage[utf8]{inputenc}
\usepackage{polski}
\usepackage{lastpage}
\usepackage{fancyhdr}

% numery strony "x / lastpage"
\pagestyle{fancy}
\fancyhf{}
\cfoot{\thepage \hspace{1pt} / \pageref{LastPage}}
% żeby też było na pierwszej stronie
\fancypagestyle{plain}{
	\fancyhf{}
	\cfoot{\thepage \hspace{1pt} / \pageref{LastPage}}
}

\title{Symulacja gry w życie Conowaya: \texttt{clife}}
\author{
	Hubert Kalbarczyk\\
	\and
	Aliaksei Samovich\\
}
\date{\today}

\begin{document}

\maketitle

\section{Cel projektu}

Celem naszego projektu jest umożliwienie użytkownikowi symulacji gry w życie Conowaya.
Ma on mieć możliwość \textbf{stworzenia konfiguracji} (rozmiar planszy, ustawienia startowe komórek itd.) według, której program zacznie symulację.
Ma także móc \textbf{zapisać wynik} i poszczególne kroki jako sekwencje obrazów.
Sama symulacja ma mieć tryb automatyczny i \textbf{tryb \textit{krok po kroku}}, gdzie użytkownik będzie ręcznie kontynuował każdą generację.
Podczas pracy program będzie też wyświetlał każdą generację w konsoli jako planszę znaków "X" i "O", odpowiednio "martwa" i "żywa".

\section{Dane wejściowe}

By program mógł zacząć działanie należy podać mu odpowiednio sformatowany plik z konfiguracją.
Ma ona być podana w pliku TOML w sekcji [Config].

\begin{itemize}

\item
Wymiary planszy podane w ilości komórek.
Plansza ma punkt [ 0, 0 ] w lewym górnym rogu i punkt [ x, y ] w prawym dolnym.
Oznacza to, że plansza 50 x 50 ma w współrzędne od [ 0, 0 ] do [ 49, 49 ].

Format: dimensions = [ x, y ]

\item
Pola, które mają być aktywowane w generacji 0, czyli na starcie.
Podane w postaci punktów w wektorze.
Punkty nie znajdujące się na planszy zostaną pominięte, zostanie o tym wypisany komunikat.

Format: active-tiles = [ [ $x_1$, $y_1$ ], [ $x_2$, $y_2$ ] ]

\item
Ilość generacji do symulowania podana w postaci liczby naturalnej dodatniej.
Liczby ujemne nie będą brane pod uwagę i spowodują zatrzymanie pracy programu.

Format: generations = n

\end{itemize}
Przykładowy plik konfiguracyjny tworzący planszę 50 na 50 komórek z Gliderem lecącym przez 25 generacji.

\begin{verbatim}
[Config]
dimensions = [ 50, 50 ]
active-tiles = [ [ 1, 0 ], [ 2, 1 ], [ 0, 2 ], [ 1, 2 ], [ 2, 2 ] ]
generations = 25
\end{verbatim}

\section{Argumenty wywołania programu}

Następujące argumenty wywołania są akceptowane przez program \texttt{clife}:

\begin{itemize}

\item
\texttt{--config path} przy tym parametrze należy podać ścieżkę do pliku z konfiguracją w formacie TOML.
Pod \textit{path} należy podstawić ścieżkę, najlepiej objętą symbolami " " aby uniknąć błędów.
Bez podania tego argumentu program zwróci błąd i przerwie działanie.
Jeżeli ścieżka będzie niepoprawna to program także zwróci błąd i przerwie działanie.

\item
\texttt{--save path} pozwala na wybór ścieżki zapisu.
Będą tam zapisywane konfiguracje z danej generacji oraz obrazy przedstawiające każdą generację.
Pliki będą nazwane od generacji.
Pod \textit{path} należy podstawić ścieżkę, najlepiej objętą symbolami " " aby uniknąć błędów.
Jeśli ścieżka nie zostanie odnaleziona lub będzie niepoprawna to program nie rozpocznie symulacji i pokaże błąd.

\item
\texttt{--save-gen gen\_num} pozwala na zapisanie generacji w postaci pliku TOML.
Ten plik będzie można użyć jako konfigurację innego uruchomienia programu.
Zostanie też zapisana ilość generacji pozostałych do końca.
Pod \textit{gen\_num} należy podstawić numery generacji do zapisania.
Jeżeli nie podamy żadnych generacji to zostaną zapisane wszystkie.
Jeśli podana generacja nie będzie w zakresie symulacji to zostanie pominięta.

\item
\texttt{--save-img gen\_num} pozwala na zapisanie generacji w postaci pliku png do podanego w \texttt{--save} miejsca.
Pod \textit{gen\_num} należy podstawić numery generacji do zapisania.
Jeżeli nie podamy żadnych generacji to zostaną zapisane wszystkie.
Jeśli podana generacja nie będzie w zakresie symulacji to zostanie pominięta.

\item
\texttt{--step-by-step} uruchamia program w trybie \textit{krok po kroku}.
Pozwala to na lepsze przyjrzenie się poszczególnym generacją.
Za pomocą naciśnięcia klawisza enter można przejść do kolejnej generacji.

\item
\texttt{--silent} program nie będzie wypisywał do konsoli każdej generacji przedstawionej w "X" i "O".
\end{itemize}
Przykładowe wywołanie programu:

\begin{verbatim}
clife --config konfig/glider.toml --save saves/glider/ --save-gen 5 20 40
\end{verbatim}

\section{Teoria}

Gra w życie Conowaya przedstawiana jest jako automat komórkowy w którym każda komórka może mieć 2 stany,
martwy (biały) i żywy (czarny).
Taki automat przechowuje siatkę komórek i zarządza ich stanami według ustalonych reguł.
W naszym przypadku są to 2 reguły:
\begin{enumerate}
\item
Martwa komórka otoczona przez 3 żywe komórki staje się żywa.
\item
Żywa komórka potrzebuje w otoczeniu 2 lub 3 żywe komórki, w przeciwnym wypadku staje się martwa.
\end{enumerate}
Komórki otaczające rozumiane są przez sąsiedztwo Moora.
Oznacza to, że każda komórka sąsiaduje z 8 innymi.
Jeśli jednak komórka leży na krawędzi planszy to przestrzeń poza nią jest uznawana za martwą.

\section{Komunikaty błędów}

Program wyświetla następujące komunikaty o błędach:
\begin{itemize}
\item
Niepoprawna ścieżka zapisu/konfiguracji:
Program nie znalazł podanej przez nas ścieżki lub nie była ona folderem.
W tym wypadku nie będzie kontynuowana praca programu.

\item
Niepoprawny argument:
Któryś z podanych przez nas argumentów wywołania nie został rozpoznany

\item
Niepoprawnie sformatowana konfiguracja:
Ten komunikat pojawi się jeśli program napotka problem podczas przetwarzania podanej przez nas konfiguracji.
Należy się upewnić, że jest ona w pliku .toml oraz zawiera wszystkie elementy wymienione w sekcji 2.

\end{itemize}

\end{document}