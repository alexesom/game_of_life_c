\documentclass[12pt]{article}
\usepackage[utf8]{inputenc}
\usepackage{polski}
\usepackage{lastpage}
\usepackage{fancyhdr}
\usepackage{forest}
\graphicspath{ {./spr_img/} }

% numery strony "x / lastpage"
\pagestyle{fancy}
\fancyhf{}
\cfoot{\thepage \hspace{1pt} / \pageref{LastPage}}
% żeby też było na pierwszej stronie
\fancypagestyle{plain}{
	\fancyhf{}
	\cfoot{\thepage \hspace{1pt} / \pageref{LastPage}}
}

\title{Sprawozdanie końcowe z projektu Gra w życie}
\author{Aliaksei Samovich, Hubert Kalbarczyk}
\date{Kwiecień 2021}

\begin{document}

\maketitle
\pagenumbering{gobble}
\newpage
\pagenumbering{arabic}

\section{Cel projektu}

\paragraph{Implementacja Gry w życie}

Naszym podstawowym celem była implementacja podstawowej logiki gry w życie z standartowymi zasadami.

\subparagraph{\large Zasady Standardowe}
\begin{itemize}
\item 
Sąsiedztwo Moore’a
    
\item
Martwa komórka, która ma dokładnie 3 żywych sąsiadów, staje się żywa w następnej jednostce czasu (rodzi się)
    
\item
Żywa komórka z 2 albo 3 żywymi sąsiadami pozostaje nadal żywa; przy innej liczbie sąsiadów umiera (z „samotności” albo „zatłoczenia”)
    
\end{itemize}

\subparagraph{\large Ustawienia obowiązkowe}
\begin{itemize}
\item
Wyświetlenie kolejnych generacji w terminale

\item 
Możliwość podania argumentów wywołania przy uruchomieniu w terminale

\item
Możliwość zapisywania podanych generacji do pliku

\item
Możliwość wywołania programu Step-by-Step i Silent

\end{itemize}

\subparagraph{\large Ustawienia dodatkowe}
\begin{itemize}
\item 
Możliwość zapisywania podanych generacji w postaci obrazku (.jpg albo .png) 

\item
Możliwość zapisywania procesu gry w postaci .gif

\item
Stworzenie pliku konfiguracyjnego dla wygodnego odczytywania potrzebnych generacji

\end{itemize}
    

\section{Dane wejściowe}

By program mógł zacząć działanie należy podać mu odpowiednio sformatowany plik z konfiguracją.
Ma ona być podana w pliku TOML w sekcji [Config].

\newpage

\section{Argumenty wywołania}

Następujące argumenty wywołania są akceptowane przez program \texttt{clife}:

\begin{itemize}

\item
\texttt{--config path} przy tym parametrze należy podać ścieżkę do pliku z konfiguracją w formacie TOML.
Pod \textit{path} należy podstawić ścieżkę, najlepiej objętą symbolami " " aby uniknąć błędów.
Bez podania tego argumentu program zwróci błąd i przerwie działanie.
Jeżeli ścieżka będzie niepoprawna to program także zwróci błąd i przerwie działanie.

\item
\texttt{--save path} pozwala na wybór ścieżki zapisu.
Będą tam zapisywane konfiguracje z danej generacji oraz obrazy przedstawiające każdą generację.
Pliki będą nazwane od generacji.
Pod \textit{path} należy podstawić ścieżkę, najlepiej objętą symbolami " " aby uniknąć błędów.
Jeśli ścieżka nie zostanie odnaleziona lub będzie niepoprawna to program nie rozpocznie symulacji i pokaże błąd.

\item
\texttt{--save-gen gen\_num} pozwala na zapisanie generacji w postaci pliku TOML.
Ten plik będzie można użyć jako konfigurację innego uruchomienia programu.
Zostanie też zapisana ilość generacji pozostałych do końca.
Pod \textit{gen\_num} należy podstawić numery generacji do zapisania.
Jeżeli nie podamy żadnych generacji to zostaną zapisane wszystkie.
Jeśli podana generacja nie będzie w zakresie symulacji to zostanie pominięta.

\item
\texttt{--save-img gen\_num} pozwala na zapisanie generacji w postaci pliku png do podanego w \texttt{--save} miejsca.
Pod \textit{gen\_num} należy podstawić numery generacji do zapisania.
Jeżeli nie podamy żadnych generacji to zostaną zapisane wszystkie.
Jeśli podana generacja nie będzie w zakresie symulacji to zostanie pominięta.

\item
\texttt{--step-by-step} uruchamia program w trybie \textit{krok po kroku}.
Pozwala to na lepsze przyjrzenie się poszczególnym generacją.
Za pomocą naciśnięcia klawisza enter można przejść do kolejnej generacji.

\item
\texttt{--silent} program nie będzie wypisywał do konsoli każdej generacji przedstawionej w postaci pola które składa się z białych i czarnych komórek.

\item
\texttt{--save-gif} zapisuje gif z animacją symulacji w podanej ścieżce
\end{itemize}
\subsection{Przykłady uruchomienia programu}
Przykładowe wywołanie programu:

\texttt{./game.out --config ./tests/glider\_gun.toml --save ./src/result\_files/ --save-gen "5 20 40" --save-gif}

Efektem tego wywołania będzie powstanie w miejscu zapisu 3 konfiguracji pozwalających na  kontynuację symulacji od danego momentu.
Oraz gif zawierający całą symulację.

\begin{figure}
\centering
\includegraphics[scale=3]{img_60}
\caption{60 klatka gifa przedstawiającego symulację}
\centering
\end{figure}

\section{Drzewo katalogów}

\begin{forest}
  for tree={
    font=\ttfamily,
    grow'=0,
    child anchor=west,
    parent anchor=south,
    anchor=west,
    calign=first,
    edge path={
      \noexpand\path [draw, \forestoption{edge}]
      (!u.south west) +(7.5pt,0) |- node[fill,inner sep=1.25pt] {} (.child anchor)\forestoption{edge label};
    },
    before typesetting nodes={
      if n=1
        {insert before={[,phantom]}}
        {}
    },
    fit=band,
    before computing xy={l=15pt},
  }
[2021L\_JIMP2\_proj\_zesp\_gr14
  [src
    [gifenc]
    [result\_files]
    [stb\_image]
    [toml]
  ]
  [bin]
  [tests]
]
\end{forest}

\paragraph{\large Szczegóły}
\begin{itemize}
    \item 
    \textbf{2021L\_JIMP2\_proj\_zesp\_gr14} - główny katalog w którym są niezbędne katalogi projektowe takie pliki jak Makefile, .gitignore, specyfikację, plik wykonawczy (po montażu) i inne

    \item
    \textbf{src} - katalog w którym są wszystkie moduły implementacyjne i katalogi które są niezbędne dla wywołania programu
    
    \begin{itemize}
        \item 
        \textbf{gifenc} - katalog z biblioteką implementującej zapisywanie do .gif przy podaniu tablicy jednowymiarowej komórek
        
        \item
        \textbf{result\_file} - katalog w którym są wszystkie pliki wynikowe po wywołaniu programu (pliki tekstowe (.txt), obrazki (.jpg, .png), animacja (.gif))
        
        \item
        \textbf{stb\_image} - katalog z biblioteką z funkcjami dla zapisywania sformatowanej w odpowiedny sposób tablicy jednowymiarowej w obrazek .png albo .jpg
        
        \item
        \textbf{toml} - katalog z biblioteką niezbędną dla implementacji parsera configuracyjnego 
    \end{itemize}
    
    \item
    \textbf{bin} - katalog w którym są przechowywane czasowe pliki obiektowe (tz. pliki z rozszerzeniem .o) niezbędne dla kompilacji w plik wykonawczy
    
    \item
    \textbf{tests} - katalog w którym są testowe pliki konfiguracyjne
    
\end{itemize}
\section{Komunikaty błędów}

Program wypisuje do konsoli komunikaty o błędach w czytelnej postaci.
Czyli np. jeżeli nie uda mu się odczytać jakiejś wartości z Config albo z podanych argumentów to powinien nas o tym poinformować i podać co się nie udało.

\section{Testy}
Testy naszego programu niestety wymagają dodania w main ponieważ nie są skierowane do użytkownika.
Są to specjalne funkcje testujące wczytanie pliku konfiguracyjnego, odczytanie argumentów.
Patrząc na to teraz zmienilibyśmy je na automatyczne lub uruchamiane przez specjalny argument.

\section{Zmiany względem specyfikacji}
\begin{itemize}
\item
Pojawił się dodatkowy  argument wywołania \texttt{--save-gif} odpowiadający za zapisanie gifa z całą symulacją.

\item
Program wywołuje się poprzez wpisanie \texttt{game.out} a nie \texttt{clife}.

\item
Generacje w konsoli są przedstawiane jako białe prostokąty i puste miejsca a nie X i O.

\item
Struktura zawarta w specyfikacji implementacyjnej jest w dużym stopniu niezgodna z rzeczywistym stanem projektu.

\item
Istnieje możliwość zapisania symulacji do pliku gif co nie było planowane.
\end{itemize}

\section{Współpraca}
\paragraph{Systeme zażądzania}
    Zdecydowaliśmy się używać tabelę Trello żeby ułatwić współprace która ma 3 sekcję: TO DO, DONE i In Progress
\paragraph{Podzielenie pracy}
\begin{itemize}
    \item 
    Podstawowa logika gry (ewolucja generacji, obliczenie ilości sąsiedzi dla komórki, generator dla testowania)  - Aliaksei
    \item 
    Zapisywanie do obrazku oraz gif - Aliaksei
    \item
    Makefile - Aliaksei
    \item
    Wyczyszczenie pamięci - Aliaksei
    \item 
    Alokacja pamięci - Aliaksei (dla Map i Image), Hubert (dla parsera toml oraz parsera argumentów wywołania)
    \item
    Zapisywanie Configu z danej generacji - Aliaksei
    \item
    Zapisywanie Mapy do pliku tekstowego (implementacja oraz testy) - Aliaksei
    \item 
    Parser konfiguracyjny (testy, parser, wczytywanie do struktury) - Hubert
    \item 
    Parser argumentów wywołania (testy, parser, wczytywanie do struktury) - Hubert
    \item 
    Tryby Step-by-Step i Silent - Hubert
    
\end{itemize}

\paragraph{Problemy}
Mieliśmy problemy we współpracę przy robieniu Makefile dla tego, że pracujemy w różnych środowiskach deweloperskich

\section{Podsumowanie i wnioski}

Przy projektach takiej wielkości nie da się już efektywnie pracować skokowo na ostatni moment.
Przekonaliśmy się o tym sami ponieważ większość niedociągnięć i błędów w naszej pracy właśnie z tego wynikała.

Jeśli chodzi o sam temat to był on dosyć ciekawy i angażujący.
Oferował on satysfakcjonującą wizualizację naszej pracy w postaci obrazków z generacjami.

 
\textbf{Wnioski:}
 
 \begin{itemize}
 \item
 Dokładnie wykonana specyfikacja implementacyjna może zaoszczędzić wiele czasu przy pisaniu kodu.
 
 \item
 Przy większych projektach trzeba pracować regularnie a nie w skokach tak jak my to robiliśmy.
 Pomaga to z poprawianiem błędów i wymyślaniem rozwiązań.
 
 \item
 Trzeba dokładnie rozplanowywać jaka praca jest do wykonania
 \end{itemize}

\end{document}